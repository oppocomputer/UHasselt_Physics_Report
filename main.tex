\documentclass{uhphysreport} 
\usepackage{lipsum} % just for the demo, may be removed

% Titles
\title{Title (bold)}
\runningtitle{*Running title: \textit{Short title ($\mathit{< 50}$ characters inc. spaces, italic, leave empty if not needed)}}

\course{Experimentele Technieken}
\acyear{2022 - 2023}
\major{$\mathit{2^{e}}$ bachelor FYS}
\date{\today}

% Authors and affiliations
\author{Student Name (2100001) \\ Student Name (2100002)} % First author
\mentor{Mentor Name} 
\promotor{Prof. Dr. Promotor J.}


% START DOCUMENT
\begin{document}
\maketitle

\thispagestyle{fancy}
\section*{Samenvatting}
Hier komt een samenvatting met een lengte van maximaal 1 bladzijde. 
Hier is voorbeeldtekst: \lipsum[1-1]

\cleardoublepage
\tableofcontents
\clearpage


\section{Inleiding}
Leid je project in, background, doel \& onderzoeksvragen, \dots 
Zorg voor voldoende referenties zoals \cite{knuth:1984} \cite{latex2e}(Endnote/Mendeley/Zotero).

\section{Theorie}
Tussentitels zijn aangeraden per onderdeel (2.1, 2.2, \dots).
Zijn er essentiele stukken theorie die noodzakelijk zijn om je verslag te begrijpen?

\section{Experimentele opzet}
Tussentitels per techniek (3.1, 3.2, \dots).
Welke technieken zijn er gebruikt?
Zorg dat je onderzoek repliceerbaar is (maar zonder overbodige details).

\section{RESULTS}
Tussentitels per topic/techniek (4.1, 4.1.1, \dots).
Beschrijf de resultaten (interpreteer ze nog niet), wat zie je (objectief)?
Zorg dat figuren een caption hebben (incl. alle afkortingen) onder de figuur en dat je zeker in de tekst refereert naar desbetreffende figuur. 
Tabellen krijgen een caption bovenaan (zelfde principe).

\section{DISCUSSION}
Tussentitels aangeraden.
Beschrijf welke conclusies er getrokken kunnen worden uit je resultaten. 
Leg linken (zowel tussen je eigen resultaten als dingen die je vindt in de literatuur).
\npar
Ook hier is extra voorbeeldtext: \lipsum[1-1]

\section{CONCLUSION}
Kort en bondig, wat leren we hier uit? 
Wat is de outlook?


\textit{Author contributions} -- \textbf{Who did what?} 
JH and JdR conceived and designed the research. 
JH and BvH performed experiments and data analysis. 
EvS and DD provided assistance with FLIM-FRET. JH and JdR wrote the paper. 
All authors carefully edited the manuscript.

\textit{Acknowledgements} -- \textbf{Who helped you? Who do you want to thank.} 
F.E. BvH acknowledges the agency for Innovation by Science and Technology (IWT Flanders) for his doctoral fellowship. 
JH is grateful for a postdoctoral scholarship from the Research Foundation Flanders (FWO Vlaanderen). 
Prof. Em. Yves Engelborghs is thanked for providing access to the confocal microscope. 
Research was funded by grants from the BOF KU Leuven, FWO, and EU (FP7 CHAARM).

\clearpage
\section{Samenvatting}
Dit document bevat de richtlijnen voor het schrijven van practicum- en projectverslagen in de Bachelor-opleiding Fysica aan de Universiteit Hasselt. Er wordt aandacht besteed aan de inhoud, de indeling en de typografische vereisten van een verslag. Onderdelen als grafieken en foutenrekening zijn van groot belang en worden dan ook extra belicht.

\section{Inleiding}
Naast wetenschappelijk leren denken, experimenteervaardigheid verwerven, modellen leren opstellen en toetsen, … is ook het schriftelijk rapporteren hierover een belangrijke vaardigheid die je in je Bachelor-opleiding moet verwerven. Het is namelijk zo dat wetenschappers hun bevindingen met anderen moeten kunnen delen, zowel mondeling – op congressen – als schriftelijk. Dit laatste doen ze door artikels te schrijven in wetenschappelijke tijdschriften. Goede verslagen leren schrijven is dus een belangrijke stap in de richting van leren communiceren met collega-fysici en andere wetenschappers.

Een goed practicumverslag rapporteert op een correcte, wetenschappelijk verantwoorde manier welke experimenten werden uitgevoerd en welke resultaten werden verkregen. Verder wordt erin uitgelegd hoe de resultaten verwerkt en geïnterpreteerd werden (welke theorie hierbij gebruikt werd, welke formules, ...). Ten slotte worden er ook besluiten uit de proeven trokken.

In principe is het uitgangspunt dat een ander persoon aan de hand van jouw verslag de proef moet over kunnen doen met een vergelijkbaar eindresultaat. Als je twijfelt of bepaalde informatie al dan niet in het verslag moet staan, komt deze regel wel van pas. Toch zijn er heel wat meer geschreven (en ongeschreven) wetten waaraan een goede wetenschappelijke publicatie moet voldoen. In deze tekst proberen we duidelijk te maken hoe je de voorbereiding VOOR, de administratie TIJDENS en de uitwerking en verslaglegging NA een practicum moet verrichten. We geven ook aanwijzingen over het maken van belangrijke onderdelen van een verslag, zoals tabellen en grafieken.
We geven ook aanwijzingen over het maken van belangrijke onderdelen van een verslag, zoals tabellen en grafieken. We geven ook aanwijzingen over het maken van belangrijke onderdelen van een verslag, zoals tabellen en grafieken. We geven ook aanwijzingen

\clearpage

\vspace{0.5cm}
\printbibliography[title={REFERENCES}]
\newpage


\end{document}