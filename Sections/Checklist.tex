% Verander item naar checkeditem als je het puntje hebt nagekeken en je verslag eraan voldoet. Enkele puntjes zijn al aangeduid bij wijze van voorbeeld en omdat deze template er automatisch (tenzij het wordt gewijzigd)
\numberlesssection{Checklist verslagen}

\begin{checklist}
    \checkeditem
        Het verslag bestaat uit deze onderdelen, in deze volgorde (optionele delen staan tussen haakjes): titel (– inhoudsopgave) – samenvatting – inleiding – theorie – experimentele opzet – resultaten en discussie – besluit (– erkenning) – literatuur (– appendices).
    \item
        Er is een titelblad met daarop: de titel van het verslag, de namen (alfabetisch geordend) + studentennummers van de studenten, het vak + de docent, het academiejaar, de naam en eventueel het logo van de universiteit.
    \checkeditem
        Het verslag is opgesteld in A4-formaat met marges van 2,5 cm. Het lettertype is Times New Roman met lettergrootte 12. De regelafstand is 1,5.
    \checkeditem
        De hoofdtitels staan in Times New Roman, grootte 16, in vet; volgende titels staan in grootte 14, in vet; de daaropvolgende titels hebben lettergrootte 12 en staan in vet, vetcursief of cursief.
    \item
        De tekst is opgedeeld in alinea’s en er is een doorlopende paragraafnummering.
    \checkeditem
        De inhoudsopgave komt overeen met de paragraafnummers en pagina’s.
    \checkeditem
        De pagina’s zijn genummerd met Arabische cijfers
    \item
        Er staan geen voetnoten in de tekst.
    \item
        Bij informatie die is overgenomen uit boeken of andere bronnen staat een bronverwijzing. De literatuurreferenties zijn opgemaakt volgens de regels in dit document.
    \item
        De woorden ‘wij’ of ‘ik’ komen niet voor in de tekst! Handelingen staan beschreven in passieve vorm.
    \item
        Er staan geen spel- of grammaticafouten in de tekst.
    \item
        Alle afkortingen en symbolen worden bij hun eerste voorkomen gedefinieerd.
    \item
        Alle vergelijkingen in de tekst zijn genummerd.
    \item
        Afleidingen en theorie die van nut zijn voor het begrijpen van de uitgevoerde metingen en die niet te lang zijn, staan in het verslag. Lange stukken theorie of afleidingen staan in een appendix. Theorie of afleidingen die niet zelf zijn afgeleid, horen niet thuis in het verslag.
    \item
        Alle meetwaarden en afgeleide grootheden – zowel in de tekst, in tabellen en in grafieken – hebben de juiste eenheid, onzekerheid en zijn correct afgerond. 
    \item
        Voor het practica van het vak mechanica: Gedetailleerde berekeningen van fouten zijn opgenomen in een appendix.
    \item
        Alle tabellen hebben een nummer en bovenschrift dat beschrijft welke gegevens er in de tabel staan. In elke kolom staan grootheid en eenheid vermeld. Bij alle waarden (met het juiste aantal significante cijfers) staat de bijhorende onzekerheid. Er wordt in de tekst minstens één keer naar elke tabel verwezen.
    \item
        Alle figuren hebben een nummer en onderschrift; er wordt in de tekst minstens één keer naar verwezen.
    \item
        Alle grafieken hebben assen met een leesbare indeling en astitels die grootheid en eenheid vermelden. De meetpunten staan mooi verspreid in het vlak van de grafiek. Foutenbalken geven de onzekerheid op de gegevenspunten weer. In het onderschrift staat een beschrijving van de grafiek, met een legende voor de gebruikte symbolen. Er wordt in de tekst minstens één keer naar elke grafiek verwezen.
    \item
        Wanneer een hele reeks grafieken werd opgesteld die hetzelfde resultaat of hetzelfde experiment weergeven, zijn deze opgenomen in een appendix, terwijl er slechts één grafiek in de hoofdtekst van het verslag staat.
\end{checklist}